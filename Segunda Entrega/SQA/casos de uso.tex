\documentclass{article}
\usepackage{longtable}

\title{SQA - Casos de Uso SGEventos}
\begin{document}
\maketitle
\section{Defeitos Gerais}
O documento de Casos de Uso não contém requisitos não funcionais.\\\\
A parte de Casos de Uso Expandido não contém pré-condições e pós-condições.\\\\
Eu acho que faltou colocar o caso de uso de Alterar/Modificar um evento, como nome, descrição, quantidade de
convidados e localização para estar de acordo com o requisito RF\_B1.2.\\\\
Por fim recomendado manter o nome do projeto como SGEventos para todos os documentos.
\newpage
\section{Defeitos Específicos}
\begin{table}[h!]
\centering
\begin{tabular}{|c|c|c|p{7cm}|}
    \hline
    \textbf{Defeito} & \textbf{Página} & \textbf{Classe} & \textbf{Descrição} \\
    \hline
    1 & 2 & FI & 1.2 Requisito de referência (RF\_F1) não existe \\
    \hline
    2 & 2 & FI & 1.3 Requisito de referência (RF\_F2) não existe \\
    \hline
    3 & 2 & O & 1.4 e 1.5 Se um evento já possui obrigatoriamente um funcionário e buffet (RF\_B2.4),
    inserir um funcionário não seria como inserir outra despesa? ou daria pra criar um evento sem funcionário e depois inserir?\\
    \hline
    4 & 2,3 & O,A & 1.6, 1.7 Adicionar e remover itens também seria como inserir e remover uma despesa adicional? \\
    \hline
    5 & 3 & O & 1.8 Se conseguimos inserir um funcionário também será possível remover um funcionário a partir das despesas? \\
    \hline
    6 & 5 & A & 2.3 É bom manter o mesmo nome de caso de uso para ambos o expandido e alto nível \\
    \hline
    6 & 6 & O & 2.5 Esse não tem Caso de Uso de Alto Nível \\
    \hline
\end{tabular}
\end{table}
\end{document}
