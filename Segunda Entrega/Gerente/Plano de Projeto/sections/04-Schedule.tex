\section{Cronograma}
    O cronograma está dividido baseado na forma de trabalho organizada em
    Gerentes, Analista e Projetista, SQA e Codificadores: \\

    \textbf{Gerentes:} Fernando Kendi Salesi e Miguel de Campos Rodrigues Moret.
    \\
    
    \textbf{Analistas e Projetistas:} Arthur Koichi Nakao e Abigail Sayury
    Nakashima\\
    
    \textbf{SQA:} Roberto Augusto dos Santos Colatto e Paulo Sergio Campos de
    Lima.\\
    
    \textbf{Codificadores:} Roberto Augusto dos Santos Colatto e Paulo Sergio
    Campos de Lima.\\
    \\
\subsection{Gantt}
    O cronograma de Gantt foi elaborado utilizando o software Jira, em que a
    segunda parte do projeto começou em 23/09/2025 até o dia 03/11/2025:
    \begin{figure}[ht!]
        \centering
        \includegraphics[width=1\linewidth]{images/Gantt2.png}  %antes width=0.5
        \caption{Gráfico de Gantt, segunda parte do projeto na cor laranja}
    \end{figure}
    \newpage
\subsection{Rede de Tarefas}
    Segue o cronograma de rede de tarefas geral do projeto, que será feito em 42
    dias. \\
    \begin{figure}[ht!]
        \centering
        \includegraphics[width=1\linewidth]{images/RedesPrincipal.PNG}   %antes width=0.5
        \caption{Rede de Tarefas, com cedo e tarde e a folga}
    \end{figure}
    \begin{table}[ht!]
    \centering
    \begin{tabular}{|c|l|} % "l" para alinhar à esquerda as descrições
        \hline
        \textbf{Sigla} & \textbf{Tarefa} \\ \hline
        A & Documento de Requisitos \\ \hline
        B & Casos de Uso \\ \hline
        C & Casos de Uso Expandidos \\ \hline
        D & Modelo Conceitual \\ \hline
        E & Diagrama de Sequência \\ \hline
        F & Diagrama de Colaboração \\ \hline
        G & Diagrama de Classes \\ \hline
        H & Codificação \\ \hline
        SC & SQA sem defeitos \\ \hline
        S & SQA correção \\ \hline
    \end{tabular}
    \caption{Tarefas da Figura 2}
    \end{table}

